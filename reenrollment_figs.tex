% figures for inclusion in appendix of reenrollment
\section*{Figures}

\begin{centering}

\begin{figure}[h!]
  \centering
  \begin{subfigure}{0.5\linewidth}
    \includegraphics[width=\linewidth]{n500_erdos}
    \caption{Initialized as  random graph}
    \label{mm:evo1}
  \end{subfigure}%
  \begin{subfigure}{0.5\linewidth}
    \includegraphics[width=\linewidth]{n500_lop}
    \caption{Initialized with half nodes fully connected, half nodes no connection}
    \label{mm:evo2}
  \end{subfigure}%
  \caption{Evolution of degree sequences in multigraph model}
  \label{mm:evo}
\end{figure}

\begin{figure}[h!]
  \centering
  \includegraphics[width=0.8\linewidth]{n500_cpi}
  \caption{Relative error between degree sequences of CPI-accelerated simulations and plain simulations}
  \label{mm:cpi}
\end{figure}

\newpage
\begin{landscape}

\begin{figure}[h!]
  \centering
  \includegraphics[width=\linewidth]{lifting_restriction_schematic}
  \caption{Overview of restriction (red) and lifting (blue) operations}
  \label{mm:schematic}
\end{figure}

\end{landscape}


\begin{figure}[h!]
  \centering
  \includegraphics[width=0.8\linewidth]{newton_error}
  \caption{Decrease of error in coarse Newton iterations}
  \label{mm:newton}
\end{figure}

\begin{figure}[h!]
  \includegraphics[width=0.8\linewidth]{dmaps_long}
  \caption{Embedding of graphs sampled from two trajectories that started from different initial conditions. The embeddings of the two trajectories meet at the shared stationary point}
  \label{mm:dmaps2}
\end{figure}


\end{centering}