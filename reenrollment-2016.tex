%% This document should serve as a base for most mathematically
%% oriented documents. There is probably a better way to do this.

\documentclass[12pt]{article}
%\renewcommand\thesubsection{\thesection.\alph{subsection}}
\usepackage{subcaption, amsfonts, amsmath, amsthm, empheq, setspace,
  lscape}
\usepackage[draft]{graphicx}
\usepackage[toc,page]{appendix}
\newtheorem*{thm:jnf}{Jordan normal form for square matrices}
%% some new commands I have no idea how they work
% \doublespacing
\newcommand*\widefbox[1]{\fbox{\hspace{2em}#1\hspace{2em}}}
\newlength\dlf
\newcommand\alignedbox[2]{
  % Argument #1 = before & if there were no box (lhs)
  % Argument #2 = after & if there were no box (rhs)
  % Alignment sign of the line
  {
    \settowidth\dlf{$\displaystyle #1$}  
    % The width of \dlf is the width of the lhs, with a displaystyle font
    \addtolength\dlf{\fboxsep+\fboxrule}  
    % Add to it the distance to the box, and the width of the line of the box
    \hspace{-\dlf}  
    % Move everything dlf units to the left, so that & #1 #2 is aligned under #1 & #2
    \boxed{#1 #2}
    % Put a box around lhs and rhs
  }
}
%% end new commands I have no idea how they work
\newcommand\ER{Erd\H{o}s-R\'{e}nyi}
\newcommand{\K}[1]{\mathcal{K}^{#1}}
\newcommand{\Kk}{\mathcal{K}^k}
\newcommand{\Forall}{\; \forall \;}
\newcommand\numberthis{\addtocounter{equation}{1}\tag{\theequation}}
\DeclareMathOperator*{\argmin}{\arg\!\min}
% \captionsetup{labelformat=empty,labelsep=none}
\usepackage[top=1in, bottom=1in, left=1in, right=1in]{geometry}
% \setlength\parindent{0pt}
\graphicspath{ {./figs/} }
\pagestyle{plain}
\begin{document}
\title{\vspace{-1cm}Dimensionality reduction in sloppy models and equation free analysis of a multigraph model}
\author{\LARGE An overview by Alexander Holiday\vspace{3mm}\\\Large  under the supervision of\vspace{3mm}\\\LARGE  Professor Yannis Kevrekidis}

\maketitle
\begin{spacing}{2}
\begin{abstract}
  The past year has mainly been devoted to furthering our study of
  sloppy models. While this centered on developing suitable toy models
  that plainly and concisely convey our approach to the problem, we
  have also developed a novel Diffusion Maps kernel which uncovers
  both sloppy and stiff parameters, enabling us to probe different
  directions in parameter space. We are currently preparing a
  manuscript on these results, and have presented our findings both at
  the Graduate Student Symposium and at the AIChE conference. In
  addition, we submitted a paper, currently in review, based on our
  work with the evolving multigraph model.
\end{abstract}

\section*{Progress and future work}

\indent Two projects were advanced over the past year: equation-free
analysis of the dynamic multigraph model and parameter reduction in
complex models. In the former case, we augmented previous results and
submitted a paper on our conclusions, while the latter study remains
an active area of research.

\subsection*{Evolving multigraph model}

The major addition to previous work on this topic involved an
embedding of a collection of graphs drawn from this model using a
modified version of Diffusion Maps. In one case, graphs were sampled
periodically as they evolved in time. Two separate trajectories were
used with two different initial conditions, but other parameters were set
such that these two distinct systems evolved to a common stationary
state. This is exhibited in Fig. (\ref{fig:dmaps-results}), which shows a
Diffusion Map of the two trajectories. Initially, their dissimilar
initial conditions lead them to be embedded far from each other, but
as they evolve towards the shared stationary state, the embeddings
overlap. \\

\begin{figure}[h!]
  \vspace{-5mm}
  \centering
  \begin{subfigure}{0.49\textwidth}
    \centering
    \includegraphics[width=\textwidth]{dmaps-3d-convergence-a}
    \subcaption{\label{fig:dmaps-results-regular}}
  \end{subfigure} %
  \begin{subfigure}{0.49\textwidth}
    \centering
    \includegraphics[width=\textwidth]{dmaps-3d-convergence-zoom-a}
    \subcaption{\label{fig:dmaps-results-zoom}}
  \end{subfigure}%
  \caption{DMAP of motif-based embeddings from two separate simulation
    trajectories: (a) embedding of two trajectories on different
    timescales. While the trajectories are initially embedded in
    separate regions due to their different initial conditions,
    they're eventually mapped close together as they evolve to the
    same coarse stationary state. Final states are circled, and point
    size grows as more steps are taken in each trajectory. (b)
    Enlarged view of the final points in the previous plot, revealing
    the approach to a similar final state. Here, the average of the
    final fifty points is circled. Due to the stochastic nature of
    the system, the final embeddings randomly fluctuate about the
    shared stationary state. \label{fig:dmaps-results}}
\end{figure}

\noindent The second case involved a collection of graphs at different
stationary states. Theory predicts that this high-dimensional ensemble
is described by just two model
parameters \cite{rath_multigraph_2012, rath_time_2012}. Fig. (\ref{fig:dmaps-ensemble}) shows a Diffusion Map of
this dataset, and illustrates its ability to uncover this hidden
low-dimensional description. \\

\begin{figure}[h!]
  \vspace{-5mm}
  \centering
  \begin{subfigure}{0.49\textwidth}
    \centering
    \includegraphics[width=\textwidth]{dmaps-2d-rho-a}
    \subcaption{\label{fig:dmaps-rho}}
  \end{subfigure} %
  \begin{subfigure}{0.49\textwidth}
    \centering
    \includegraphics[width=\textwidth]{dmaps-2d-kappa-a}
    \subcaption{\label{fig:dmaps-kappa}}
  \end{subfigure}%
  \caption{DMAP of motif-based embeddings from collection of different
    simulations run with different parameter values: (a) coloring the
    two-dimensional DMAPS embedding with $\rho$ and (b) coloring the
    two-dimensional DMAPS embedding with $\kappa$. As with PCA, DMAPS
    uncovered the variables governing the stationary state, known from
    Bal\'{a}zs' paper \label{fig:dmaps-ensemble}}
\end{figure}


\noindent In addition to these new findings, we achieved improvements in the
stability of our previous equation-free techniques. This followed from
the development of a more constrained projection operator in our
coarse-projective integration scheme which ensures that the total
number of edges is conserved throughout the simulation.

\subsection*{Model sloppiness}

Our research into model sloppiness has progressed steadily, and we've succesfully developed a variation of Diffusion Maps that captures the significant parameter combinations instead of the sloppy, ill-constrained ones. While the normal implementation of the method would use a kernel of the form

\begin{align}
  k(\theta_i, \theta_j) = \exp(-\frac{\| \theta_i - \theta_j\|^2}{\epsilon^2})
\end{align}

\noindent where $\theta_i$ is the $i^{th}$ set of parameters \cite{coifman_diffusion_2006}, by expanding it to include the model predictions at $\theta_i$, denoted $\mu(\theta_i)$, we can create a sort of anisotropic diffusion that prefers directions along which model predictions vary \cite{lafon_stephane_diffusion_2004}. This kernel takes the form

\begin{align}
  k(\theta_i, \theta_j) = \exp(-\frac{\| \theta_i - \theta_j\|^2}{\epsilon^2} - \frac{\| \mu(\theta_i) - \mu(\theta_j)\|^2}{\lambda^2})
\end{align}

\noindent and $\lambda$ is an additional parameter whose effect is similar to that of $\epsilon$ in the sense that it defines a meaningful neighborhood for each prediction $\mu(\theta_i)$. \\

\noindent As we continued, it became increasingly apparent that viewing sloppiness through the lens of optimization was unproductive, and that it should instead be considered in terms of the more basic interaction between parameter space and actual model predictions. This concept is illustrated in Fig. (\ref{fig:mm}) in which different regions of model space (top figures) are mapped to wildly different volumes of parameter space (bottom figures). In this particular case, sloppiness arises as a byproduct of a singular perturbation in the system such that for any $1/ \epsilon$ large enough, the model predictions remain nearly constant. Thus the large black region of parameter space in the lower figure maps to a relatively small surface in model-prediction space, while the small ellipse in parameter space maps to a relatively large surface. \\

\begin{figure}[h!]
  \vspace{-5mm}
  \centering
  \includegraphics[width=\textwidth]{mm-fig-v3}
  \caption{Illustration of sloppiness arising in singularly perturbed system. Model predictions (top) map to certain parameter comibinations (bottom). However, if a set of parameters correspond to a singularly perturbed system ($1/\epsilon$ large in this case) then large regions of parameter space are mapped to a very small region in model prediction space. \label{fig:mm}}
\end{figure}


\noindent There are many directions for future work. \textbf{First, we would like to develop a method of efficiently sampling parameter space.} By using a sort of directed Monte Carlo method, we believe we can guide the sampling procedure to quickly cover a certain region of parameter space: a sloppy surface, for instance. \textbf{We could then combine this sampling routine with our anisotropic DMAPS kernl to optimize models in DMAPS space, instead of in the sloppy parameter space.} This would also require an applicatin of geometric harmonics \cite{coifman_geometric_2006, chiavazzo_reduced_2014} to translate between Diffusion Map coordinates and model parameters, but would be an interesting application of many different dimensionality-reduction techniques. \textbf{Finally, we would like to apply these methods to real-world systems \cite{boender_risk_2007}.} Preliminary analysis of Prof. Brynildsen's data performed several months ago seemed promising, and more could be done at this stage.

\end{spacing}

% % figures for inclusion in appendix of reenrollment
\section*{Figures}

\begin{centering}

\begin{figure}[h!]
  \centering
  \begin{subfigure}{0.5\linewidth}
    \includegraphics[width=\linewidth]{n500_erdos}
    \caption{Initialized as  random graph}
    \label{mm:evo1}
  \end{subfigure}%
  \begin{subfigure}{0.5\linewidth}
    \includegraphics[width=\linewidth]{n500_lop}
    \caption{Initialized with half nodes fully connected, half nodes no connection}
    \label{mm:evo2}
  \end{subfigure}%
  \caption{Evolution of degree sequences in multigraph model}
  \label{mm:evo}
\end{figure}

\begin{figure}[h!]
  \centering
  \includegraphics[width=0.8\linewidth]{n500_cpi}
  \caption{Relative error between degree sequences of CPI-accelerated simulations and plain simulations}
  \label{mm:cpi}
\end{figure}

\newpage
\begin{landscape}

\begin{figure}[h!]
  \centering
  \includegraphics[width=\linewidth]{lifting_restriction_schematic}
  \caption{Overview of restriction (red) and lifting (blue) operations}
  \label{mm:schematic}
\end{figure}

\end{landscape}


\begin{figure}[h!]
  \centering
  \includegraphics[width=0.8\linewidth]{newton_error}
  \caption{Decrease of error in coarse Newton iterations}
  \label{mm:newton}
\end{figure}

\begin{figure}[h!]
  \includegraphics[width=0.8\linewidth]{dmaps_long}
  \caption{Embedding of graphs sampled from two trajectories that started from different initial conditions. The embeddings of the two trajectories meet at the shared stationary point}
  \label{mm:dmaps2}
\end{figure}


\end{centering}

\bibliographystyle{abbrv}
\bibliography{./reenrollment-2016}


\end{document}

